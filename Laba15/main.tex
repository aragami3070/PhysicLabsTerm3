\documentclass[bachelor, och, referat, times]{SCWorks}
% параметр - тип обучения - одно из значений:
%    spec     - специальность
%    bachelor - бакалавриат (по умолчанию)
%    master   - магистратура
% параметр - форма обучения - одно из значений:
%    och   - очное (по умолчанию)
%    zaoch - заочное
% параметр - тип работы - одно из значений:
%    referat    - реферат
%    coursework - курсовая работа (по умолчанию)
%    diploma    - дипломная работа
%    pract      - отчет по практике
%    pract      - отчет о научно-исследовательской работе
%    autoref    - автореферат выпускной работы
%    assignment - задание на выпускную квалификационную работу
%    review     - отзыв руководителя
%    critique   - рецензия на выпускную работу
% параметр - включение шрифта
%    times    - включение шрифта Times New Roman (если установлен)
%               по умолчанию выключен 

\usepackage[T2A]{fontenc}
\usepackage[utf8]{inputenc}
\usepackage{graphicx}
\usepackage[sort,compress]{cite}
\usepackage{amsmath}
\usepackage{amssymb}
\usepackage{amsthm}
\usepackage{fancyvrb}
\usepackage{longtable}
\usepackage{array}
\usepackage{makecell}
\usepackage{multirow}
\usepackage[english,russian]{babel}

\usepackage{tempora}
\usepackage[hidelinks]{hyperref}

\usepackage{pgfplots}
\usepackage{tikz}
\usepackage{float}
\pgfplotsset{compat = newest}

\usepackage{minted}
\setminted[c++]{linenos, breaklines = true, style = bw, fontsize = \small}
\usepackage{forloop}
\usepackage{pgffor}

\begin{document}


% Кафедра (в родительном падеже)
\chair{математической кибернетики и компьютерных наук}

% Тема работы
\title{Отчетная работа}

% Курс
\course{1}

% Группа
\group{251}

% Факультет (в родительном падеже) (по умолчанию "факультета КНиИТ")
%\department{факультета КНиИТ}

% Специальность/направление код - наименование

\napravlenie{09.03.04 "--- Программная инженерия}

% Для студентки. Для работы студента следующая команда не нужна.
\studenttitle{студентов}
% Фамилия, имя, отчество в родительном падеже
\author{Смирнова Егора и Храмова Александра}

% Год выполнения отчета
\date{2024}

\maketitle

% Включение нумерации рисунков, формул и таблиц по разделам
% (по умолчанию - нумерация сквозная)
% (допускается оба вида нумерации)
%\secNumbering

\tableofcontents

\foreach \n in {1,...,3}{
    \input{sections/section \n}
}


\end{document}
